\documentclass{scrartcl}
% \usepackage[latin1]{inputenc}
% \usepackage[T1]{fontenc}
\usepackage{amsthm, amssymb, amsmath}
\usepackage[french]{babel}
\usepackage{hyperref}
\usepackage{braket}

\theoremstyle{plain}
\newtheorem{theoreme}{Théorème}[section]
\newtheorem{proposition}[theoreme]{Proposition}
\newtheorem{corollaire}[theoreme]{Corollaire}
\newtheorem{lemme}[theoreme]{Lemme}
\newtheorem{exo}{Exercice}
\theoremstyle{definition}
\newtheorem{definition}[theoreme]{Définition}
\theoremstyle{remark}
\newtheorem*{remarque}{Remarque}

\begin{document}
\section{Rappels}
\begin{definition}
On appelle bla bla bla bla bla bla bla bla bla bla bla bla bla bla bla
bla bla bla bla bla bla bla bla bla.
\end{definition}
\begin{proposition} Si bla bla bla bla bla bla bla bla bla bla bla bla alors bla bla bla bla
bla bla bla bla.
\end{proposition}
\begin{proof}
On commence par prouver que bla bla bla bla bla bla bla bla bla bla
bla bla bla bla bla bla bla bla bla bla bla bla bla.
\end{proof}
\begin{corollaire}
Si bla bla bla bla bla bla bla bla bla bla bla bla alors bla bla bla bla bla
bla bla bla.
\end{corollaire}
\begin{exo}
Montrer que bla bla bla bla bla bla bla bla bla bla bla bla bla bla bla bla
bla bla bla bla bla bla bla.
\end{exo}
\begin{exo}
Montrer que bla bla bla bla bla bla bla bla bla bla bla bla bla bla bla bla
bla bla bla bla bla bla bla.
\end{exo}
\section{Approfondissements}
\begin{definition}
On appelle bla bla bla bla bla bla bla bla bla bla bla bla bla bla bla
bla bla bla bla bla bla bla bla bla.
\end{definition}
\begin{lemme}
Lemme 2.2. Si bla bla bla bla bla bla bla bla bla bla bla bla alors bla bla bla bla bla bla
bla bla.
\end{lemme}
\begin{proof}
Démonstration. On montre que bla bla bla bla bla bla bla bla bla bla bla bla bla bla
bla bla bla bla bla bla bla bla.
\end{proof}
\begin{theoreme}
 Si bla bla bla bla bla bla bla bla bla bla bla bla alors bla bla bla bla bla
bla bla bla.
\end{theoreme}
\begin{proof}
On commence par montrer que bla bla bla bla bla bla bla bla bla bla
bla bla bla bla bla bla bla bla bla bla bla bla.
\end{proof}
\begin{remarque}
 Remarquons que bla bla bla bla bla bla bla bla bla bla bla bla bla bla bla
bla bla bla bla bla bla bla bla.
\end{remarque}
\begin{exo}
Exercice 3. Montrer que bla bla bla bla bla bla bla bla bla bla bla bla bla bla bla bla
bla bla bla bla bla bla bla.
\end{exo}

Soit $f$ une function verifiant
\begin{equation}\label{eq.f}
f(x) = 2x+1
\end{equation}
On a $f(x) - 1 = 2x$ d'apres la formule~\eqref{eq.f}.
\begin{equation}
  {(x^{2})}^{3} = x^{2^{3}}
\end{equation}
\begin{equation}
  F_{n}=7^{2^{n}}+ 1
\end{equation}
\[
  y = x^{2} \iff x = y ^{1/2}
\]
\[
  x > 0 \implies x^{2} \neq 0
\]
\[
  x \in X \setminus Y \implies x \not \in Y
\]
\[
  u_{n+1}=\sqrt[3]{1+x}
\]
\[
  x_{5} = \sqrt{1+\sqrt{2+\sqrt{3+\sqrt{4+\sqrt{5}}}}}
\]
\[
  x^{1/3} = x^{\frac{1}{3}} = \sqrt[3]{x}
\]
\[
  \sqrt{2} = 1 + \frac{1}{2+\frac{1}{2+\frac{1}{2+\frac{1}{\ddots}}}}
\]
\[
  \frac{\pi^{2}}{6} + \gamma = \Gamma(n) + \sqrt[n]{1+\alpha}
\]
\[
  (\sqrt{x})^{2} = x \quad \text{mais}\quad \sqrt{x^{2}} \neq x \quad\text{en general}
\]
\[
  \cos^{2}+sin^{2} = 1
\]
\[
  2^{\ln(x)} = x^{\ln(2)}
\]
\[
  \sum_{n=1}^{+\infty}\frac{1}{n^{2}} = \frac{\pi^{2}}{6}
\]
\[
  \int_{0}^{1}-\frac{\ln(1-t)}{t}dt\approx1,64493
\]
\[
  \max_{\substack{x, y \in E\\x \cdot y}} = \varphi(x)
\]
\[
  \overrightarrow{OM} = \underbrace{O + \overrightarrow{u}}_{point+vecteur}
\]
\[
  \lVert x \rVert = 1 \iff \langle x,x \rangle = 1
\]
\[
  \lvert \{ 1,2,\dots,n\} \rvert = n
\]
\[
  \lfloor x^{2}+\epsilon \rfloor = \lceil \sqrt{y} + \delta \rceil
\]
\[
  \left\lfloor\sum_{n=1}^{N}u_{n} \right\rfloor ^{2} = N^{2}+N+1
\]
\[
  \left[ 1+ \left(\int_{0}^{\sqrt{2}}f \right)^{2}\right] = \gamma
\]
\[
\Set{a+ib \in \mathbb{C} | a < b}
\]
\[
  \mathcal{L}f = \int_{a}^{b}f \, \mathrm{d}t
\]
\[
  \left[
  \begin{array}{r}
    a \in \mathbb{C} \\
   a \not \in \mathbb{R}
  \end{array}
  \right]
  \implies
  a \in \mathbb{C} \setminus \mathbb{R}
\]
\[
  \mathbb{M} =
  \begin{pmatrix}
    m_{1,1}&& \cdots && m_{1,n} \\
    \vdots && \ddots && \vdots \\
    m_{n,1} && \cdots && m_{n,n}
  \end{pmatrix}
\]
\end{document}
